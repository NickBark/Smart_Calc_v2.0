\documentclass[a4paper,12pt]{article}

\usepackage{cmap} % поиск в PDF
\usepackage[T2A]{fontenc} % кодировка
\usepackage[utf8]{inputenc} % кодировка исходного текста
\usepackage[english,russian]{babel} % локализация и переносы

\author{Tiaraurz}
\title{SmartCalc v1.0}
\date{\today}

\begin{document}
	
\maketitle
\clearpage

\section{Введение}
Программа имеет графический пользовательский интерфейс, разработанный на основе GUI-библиотеки Qt 5.15.3 в кроссплатформенной свободной среде разработки Qt Creator.

Для начала использования калькулятора необходимо выполнить установку с помощью команды $make$ $install$. Эта команда установит программу в директорию $HOME/School\_Projects\_Review/SmartCalc\_v1/$.

Вы также можете удалить приложение с вашего компьютера, введя команду $make$ $uninstall$.

Есть возможность создать архив проекта с помощью команды $make$ $dist$.

\section{Навигация}
В верхней части окна располагается menu bar, где пользователь может выбрать нужный ему тип калькулятора.
\section{Графический калькулятор}
Графический калькулятор представляет из себя смесь обычного калькулятора и графика, который строится в случае, если в формуле указана переменная x.
\section{Кредитный калькулятор}
Кредитный калькулятор позволяет заемщику рассчитать кредит. Есть возможность выбрать тип ежемесячных платежей (дифференцированные или аннуитетные).
\section{Депозитный калькулятор}
Депозитный калькулятор поможет вам быстро рассчитать проценты по любому вкладу, в том числе с капитализацией, с пополнениями и с учетом налогов.

По нажатию кнопки пополнения или кнопки снятия появится новое окно, где можно указать соответствующие частичные пополнения или частичные снятия.


	
	
\end{document}